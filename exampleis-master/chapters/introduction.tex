%!TEX root = ../username.tex
\chapter{Introduction}\label{intro}
Over the past decade, natural language processors have become more and more commonplace. In late 2011, Apple introduced their digital assistant, Siri, to iOS. Siri was the successor to voice control, an iOS feature that allowed the user to interact with their phone using their voice, though in both an extremely limited fashion and rather unreliably. Siri was powered by a much better voice recognition model. Not only could it process many commands into actions, but it could even ask for clarification based on context. Since then, Google introduced the Google Assistant to its Android and ChromeOS platforms, Amazon released smart home products powered by Alexa, and Siri has been ported to MacOS and WatchOS. All of these digital assistants rely on understanding the commands of users, and the only way this is possible is through natural language processing.                                                                                 
                                                                                                           
From Q2 2017 to Q2 2018, smart speakers, the quintessential aspect of the smart home to many, had shipments grow 187\% \cite{canalys}. This growth is expected to continue in the coming years, with smart speakers and other smart home technologies becoming an enormous industry. As this industry expands, there will be greater and greater access to these systems among the general population. Smart devices offer more than simply convenience, they could become an amazing source of educational opportunities which are not otherwise available for many. For example, what if it was possible for a student to ask their smart screen why a mathematical claim was true, and that system could respond by displaying the proof?                                                 
                                                                                                           
While smart devices do not appear in this project, we have created a front end interface for students to 
more easily investigate mathematical questions, without the barrier of technical language.               
                                                                                                           
\section{Problem Statement}                                                                           
Automated theorem proving systems are a powerful tool which can allow a student to check their work or a 
researcher to prove a claim which could be extremely time consuming if done by hand. Unfortunately, these
systems are not trivial to use, with the use of any such system requiring the user to learn the proper method for input to that system. As a result, these systems are quite intimidating to learn, and once one learns one such system, it is easy to be stuck using what is already known, even if better options may be available. A wonderful solution to this problem is leveraging computational power to convert a human language request, e.g. ``Prove that the length of the third side of a triangle is determined by the two other sides and the angle between them'' into a language the proving system understands.                     
                                                                                                           
                                                                                                           
  \subsection{Project Aims}                                                                                
This project has two separate but interconnected goals; the production of a natural language processor, and the creation of a basic automated prover. The processor:                                              
  \begin{itemize}                                                                                          
  \item{Serves as a front end for any automated prover.}                                                   
  \item{Is usable without any understanding of symbolic notation or programming.}                          
  \item{Is modular enough to become the front-end to any automated proving system.}                        
  \item{Accurately translates standard English to symbolic notation.}                                      
  \end{itemize}                                                                                            
                                                                                                           
  The prover had far more modest aims:                                                                     
  \begin{itemize}                                                                                          
  \item{Understand basic questions in a limited scope, such as geometry or arithmetic.}                    
  \item{Take symbolic input, and return the truth value of said statement.}                                
  \item{Display the steps of the proof to the end user.}                                 

  \end{itemize}

  As the two pieces are separate, the processor is not limited by the prover, and may instead be joined with a more robust system by specifying what said system is. This is necessary to convert the request properly, so that the processor knows which symbolic language to translate to.                                 
  \subsubsection{Natural Language Processor}                                                               
The processor will takes a user's input in English, and determines the goal of the user based on the content of that input. The user is not required to know anything, other than that the system is intended for mathematical queries and will be able to do nothing else. Once the input is analyzed, the processor converts it to the proper symbolic language for the automated prover.                                         
  \subsubsection{Automated Theorem Prover}                                                                 
                                                                                                           
  The prover takes a mathematical claim, in a symbolic notation, and applies axioms, theorems, and other knowledge as necessary to prove or disprove the claim. If the prover is able to use theorems in its proofs, it is barred from using them in such a way as to give the appearance of circular logic. For example, when a user asks for a proof of the Pythagorean Theorem, the prover may not use the Pythagorean Theorem as its justification for the truth of the theorem. %We can do this by preventing the prover from using single step proofs except in the case of an appeal to an axiom!                                                
Should the user ask for a proof of an axiom, it is permissible to simply point to the axiom. The prover is not expected to explain an axiom to a user.   
