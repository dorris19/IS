%!TEX root = ../username.tex
\chapter{Conclusion}

This study serves as a proof-of-concept for the idea of a conversational theorem proving system more than it stands as a successful translator in its own right. The study has shown that, even with basic techniques to parse language and a limited domain of questions, the program can still be useful for testing simple claims. More sophisticated language parsing systems, along with a more in depth understanding of automated theorem proving programs, could make a robust, easy to use system for both students and mathematicians.

\section{Failings}

The original goal for this project was to produce a parser and automated theorem prover. While a translator was created with the assistance of Stanford's CoreNLP server, an automated theorem prover was never produced. This was due to time constraints on the project as well as the complexity of doing so. Completing the parser took the majority of the year, the reasons for which will be discussed shortly. Were the entire period spent on creating the prover, it is possible that a slow, limited prover would have been produced.

Progress on software was slowed primarily due to initially overly specific software design and difficulties with TPTP formatting. Initially, progress on the software appeared to be advancing very quickly as the problem domain was only that of set theory. The questions the software can handle are of the form, ``Is some $X$ in $Y$?''\ expanded out based on relations between $X$, $Y$, and other sets. Unfortunately, having only a single tester during the design of the software led to a rigid structure for queries. Due to personal preference, a standard query would never be of the form ``Does $Y$ have some element $X$?'' This led to easy design; if the term ``in'' or ``element'' were spotted, then the two terms within a certain distance could be taken as the variables which were being acted upon. This design, though, is not able to be generalized to new domains. Parts of speech were initally ignored, with the design based upon the general structure of these set theory queries. Once algebraic tests were to be performed, functionality immediately broke down. Initial changes were to continue adding more rules to a flawed system, rather than throwing everything out to create a proper foundation. This slowed down the progress which could have been made over several weeks to improve the parser.

Issues with TPTP and its relationship with Vampire are an area which seems to have problems on all sides, in many ways due to the target audience. The expected users for TPTP and Vampire are not undergraduate students, and do not appear to be graduate students either. They are designed with an expectation of knowledge that someone trained in using automated theorem proving systems would have, as well as those with experience with using first order logic. Vampire is simple to use, given that proper TPTP files are given to it, but it is not particularly clear what a good file is. Vampire intends to have a reference manual, but one has yet to be created. So, users are instead pointed to a 2013 academic paper on the software. The paper is under 15 pages in length, so it does not have the space to go into depth on what files should look like beyond more than very rudimentary examples. The TPTP webpage contains many dead links and necessarily vague instructions to ensure compatibility with a wide range of theorem provers. Personal lack of knowledge on the subject is a failing on the part of the author, but it seems only fair to have addressed the many unintentional hurdles put in place by a lack of clear documentation for both TPTP and Vampire.

\section{Future Work}

The most immediately clear expansion for this idea would be the incorporation of language analysis techniques which use machine intelligence. The issues faced here regarding more particular word choice, the addition of ``and'' between all clauses, and the ban on postfix function words could be eliminated. As a goal of this study was ease of use, being more accepting of language would be a great improvement.

A second avenue would be an expansion of subject domains. This could be accomplished through the addition of more axioms for further subjects, but simply because it requires no new technology does not mean the task is easy. To define anything in first order logic requires much more time than it appears to, especially as some claims must be rephrased to become several claims.

Cleaning up output to a human-readable form would be an amazing addition to the software and to theorem proving systems in general. While the provers helpfully put out what operations are performed in order to determine the truth value of a conjecture, this output is far from simple to read. Processing steps appear with what seem to be arbitrary numbers alongside them, refering to rules which a user may not have even included as reasoning for the action performed. If this output could be analyzed to provide greater clarity to what the prover has done, it would be possible to learn techniques from the program rather than simply being given a truth value.

The final addition which could make this software reach its full potential would be connection to digital assistants. With an always-on server ready to process input and run said input through Vampire, students could quickly check their work or answer questions they find interesting. Ideally, some sort of system would be put in place to prevent these results from being used to cheat on academic assignments, while still granting enough information for a student to follow the progression of the proof.
