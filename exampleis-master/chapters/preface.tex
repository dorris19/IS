%!TEX root = ../username.tex
\chapter*{Preface}\label{pref}
\addcontentsline{toc}{chapter}{Preface}
\lettrine[lines=2, lhang=0.33, loversize=0.1]{T}he purpose of this document is to provide you with a template for typesetting your IS using \LaTeX\index{LaTeX@\LaTeX}. \lt is very similar to HTML in the sense that it is a markup language. What does this mean? Well, basically it means you need only enter the commands for structuring your IS, i.e., identify chapters, sections, subsections, equations, quotes, etc. You do not need to worry about any of the formatting. The  \texttt{woosterthesis} class takes care of all of the formatting.

Here is how I plan on introducing you to \LaTeX. The Introduction gives some reasons for why one might find \lt superior to MS Word\texttrademark. Chapter \ref{text} will demonstrate how one starts typesetting a document and works with text in \LaTeX. Chapter \ref{graphics} discusses the creation of tables and how one puts figures into a thesis. Chapter \ref{bibind} talks about creating a bibliography/references section and an index. There are three Appendices which discuss typesetting mathematics and computer program code. The Afterword will discuss some of the particulars of how a \lt document gets processed and what packages the \texttt{woosterthesis} class uses and are assumed to be available on your system.

Hopefully, this document will be enough to get you started. If you have questions please refer to \citet{mgbcr04,kd03,ophs03,feu02,fly03}, or \citet{gra96}. 