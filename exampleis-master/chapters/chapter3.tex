%!TEX root = ../username.tex
\chapter{Software Implementation and Use}

\section{Requirements for Use}

To use the software created during this project, the user must have an installation of Python 3, as well as the following packages:
\begin{itemize}
\item requests
\item NLTK
\end{itemize}

The user will also need to run the Stanford CoreNLP server, which can be obtained from \url{https://github.com/stanfordnlp/CoreNLP}, as well as the Vampire theorem prover, available at \url{https://github.com/vprover/vampire}. Vampire is to built on the release version, using the included documentation to do so. The CoreNLP server need merely to be decompressed by the user, and then within the directory the command
\begin{verbatim}
java -mx4g -cp "*"
edu.stanford.nlp.pipeline.StanfordCoreNLPServer
-port 9000 -timeout 15000
\end{verbatim}
which will set a timeout of 15000 milliseconds for processing of any statement sent to it.

\section{treeRead}

In order to use the data output by the CoreNLP server, a simple API was created which allows for all of the necessary information for conversion from English to TPTP format. The bulk of the module is within the \texttt{children() } function.

The \texttt{children() } function implements what is essentially a Lisp parser for the Stanford tree. After being provided with a tree to parse and a starting index, the children of the given tree are returned. As the number of children can be any value of one or greater, the function determines where children end by keeping track of parentheses, adding a child to a list of children whenever the count of parentheses returns to zero. After all children are accounted for, the function returns both this list and the final position within the tree. Currently, this final position is not used due to alterations in other functions since its creation.

The functions \texttt{word()}, \texttt{find\_all()}, and \texttt{check\_not()} provide utility for the functions which follow. As the name implies, \texttt{word()} returns the word associated with a given part of speech tag. In order to allow for different words with the same tag to be found, there is an optional parameter \texttt{shift}, which changes the location at which the search for the part of speech tag begins. A second value is also returned by \texttt{word()}, which gives the final index within the tree which was visited.

The function \texttt{find\_all()} performs a search of the given tree for a provided string. Upon finding an instance of the string, the index it begins at is added to a list, and the search is run once more from the position immediately following the discovered string. Finally, \texttt{check\_not()} determines if, prior to the final variable which has been found, the word ``not'' was present. If so, a tilde (\sim) is added to the front of the function, indicating that that claim is not fulfilled.

\subsection{Noun Phrase Breakdown Functions}

The final three functions are \texttt{NP\_PP()}, \texttt{NP\_VP()}, and \texttt{NN()}, creatively named based on how the original noun phrase (NP) they come from is constructed. \texttt{NP\_PP()} is called when a noun phrase and prepositional phrase (PP) are the children of a noun phrase. It first locates an operator term which will be present as a child of the noun phrase, such as ``equal'' or ``element''. Following this, the child noun phrase's sibling, the prepositional phrase, is searched for two nouns (NN), each of which are the variables which the function acts upon. For example, the prepositional phrase could contain the sets $X$ and $Y$, which the operator from before could inform us indicates that $X$ is a \textit{subset} of $Y$.

\label{npvp}
\texttt{NP\_VP()} is executed when the noun phrase produces a noun phrase and verb phrase (VP). The noun phrase is searched for a variable, while the verb phrase is searched for two nouns, one of which is a function and the other of which is a variable.

Finally, \texttt{NN()} executes only when the noun phrase has a single child which is a noun. In this instance, said noun will be a variable. To determine the function and other variable, the sibling of the noun phrase is determined which will be a verb phrase containing the remaining variable(s) and function. In this case, it is possible that rather than a noun, a number (cardinal number, CD) will be a variable. This is checked for by determining if the tag ``CD'' is present in the verb phrase; if it is, then said tag is used to find a variable, while if it is not, the standard ``NN'' tag is used.

\section{How to Use the Program}

To run the program, simply place the Vampire executable in the same directory as the program. Then, run the main Python file and provide a statement to test when prompted. The sentence will then be processed and converted to TPTP format behind the scenes, returning the output from Vampire when running the input claim with the proper domain of axioms. It is important to remember here that a refutation returned by Vampire indicates a true conjecture, while any other output means no contradiction was found between the negated conjecture and provided axioms.

\subsection{Proper Language Formatting}

To allow for accurate processing of input, certain limitations exist on the natural language portion. The first and most basic change, is a requirement to separate each claim with a comma. This means that, rather than stating ``$X$ is less than $Y$ and $Y$ is less than $Z$'', input will be of the form ``$X$ is less than $Y$, and $Y$ is less than $Z$''. This is necessary not due to the implementation of the translator between trees and TPTP format, but rather due to occassional confusion during sentence parsing. Without clear separation by the comma, it is often the case that the ``and'' assigns a relationship between what is meant to be a new independent clause and the previous clause. By simply including the comma, this issue is almost always avoided.

A second alteration to the language required for using the program is an elimination of words like ``it'' which allow for indirect, non-explicit reference. This is more than just an elimination of words in this case, it is a change to some usual ways of speech. Consider the statment ``There was riot in the market, which caused it to shut down.'' This includes the aforementioned ``it'' but also uses ``which'' to avoid repetition of ``the riot.'' To put a statement like this into our program, though obviously this statement is rather light in mathematical reference, one would input ``There was a riot in the market, the riot caused the market to shutdown.'' This would be a rather awkward manner of speech, but it allows for much more clarity than indirect reference does. 

This change will be defended with one further example, as it is unfair to say it is a weakness of the system. Consider a statement like ``When Joe shook Tom's hand, he was completely unaware that he would die within the year.'' In this context, which is grammatically correct English, it is quite difficult to determine to who the ``he'' applies to in each situation. Were there to simply be a change to reusing nouns in this situation, we would easily see that Joe was unaware that Tom would be dying.

Another limitation on word choice is the naming of variables. While it is acceptable to name a variable after any noun or most capitalized characters, the names ``A'' and ``I'' are reserved. When these characters are used, they are identified as different parts of speech than other variable names, so cause unreliable functionality.

Language is also limited, currently, by positioning of operator statements, those which translate as meaningful functions. These operators \textit{must} not come after the two objects they are acting on. A statement such as ``$B$ and $C$ are not equal'' would thus be incorrect, as the language should have been ``$B$ does not equal $C$''.

The remaining changes relate to vocabulary choice for mathematical terms and directionality of statements. Consider the example of some element $X$ which is in a set $L$. While typically it does not matter if we say ``$X$ is in $L$,'' ``$X$ is an element of $L$,'' ``$L$ has an element $X$,'' and so on, it matters to the program due to the way axioms are defined and claims are constructed during translation. The translator always assumes that the set comes after the element, so the third statement from before, ``$L$ has an element $X$'', would actually translate as an element $L$ in the set $X$. The other change to this language is similar to the removal of ``it'' and other indirect vocabulary -- the membership operator is ``element'' rather than ``in'', so the term ``element'' must be used to describe the relationship.

Usage is also limited due to Vampire being a first-order theorem prover. Due to this, it is not possible to call a function on a variable, meaning statements such as ``The sum of $X$ and $Y$ equals $Z$'' are unable to be run. This is, of course, a major limitation. At this time, there does not appear to be a resolution to this problem, greatly reducing the functionality of the system for some domains, namely any dealing with arithmetic. This does not affect functionality with questions of membership or comparisons as long as they may be phrased in such a way as to avoid said conflict.

In Table \ref{Formatting}, inappropriate input sentences are shown alongside the same query written properly.

\begin{table}[h!]
\resizebox{\textwidth}{!}{%
\begin{tabular}{ll}
\multicolumn{1}{c}{Incorrect Formatting}                                                                                            & \multicolumn{1}{c}{Correct Formatting}                                                                                          \\
 & \\
\begin{tabular}[c]{@{}l@{}}``$X$ is less than $Y$ which is less\\ than $Z$ so $X$ is less than $Z$''\end{tabular}                   & \begin{tabular}[c]{@{}l@{}}``$X$ is less than $Y$, and $Y$ is less\\ than $Z$, thus $X$ is less than $Z$'\end{tabular}              \\
 & \\
\begin{tabular}[c]{@{}l@{}}``If a set $A$ has an element $B$ and $B$ is\\ not in $C$, then $C$ and $A$ are not equal''\end{tabular} & \begin{tabular}[c]{@{}l@{}}``$B$ is an element of $D$, and $B$ is not\\ an element of $C$, so $C$ does not equal $D$''\end{tabular} \\
 & \\
``The following values are not equal: $a$, $b$, $c$''                                                                               & \begin{tabular}[c]{@{}l@{}}``$B$ does not equal $C$, and $C$ does not equal $D$,\\ and $B$ does not equal $D$''\end{tabular}        \\
\end{tabular}%
}
\caption{Examples of proper and improper formatting.}
\label{Formatting}
\end{table}
\subsection{Transitional TPTP File and Vampire}

Following the parsing of the input claim, the program produces a TPTP format file containing selected axioms along with the generated conjecture. Axioms and conjectures within TPTP take the following form:

\texttt{fof(subsetDef, \textit{axiom},((}

\texttt{(subset(X, Y))}

\texttt{\& (element(A, X))}

\texttt{=> element(A, Y)))).}

\noindent
where \textit{axiom} is used for an axiom, and \textit{conjecture} for a conjecture.

Once this file has been created, Vampire is run on it. The program then outputs the results of the test, displaying either the successful run which details whether or not the conjecture was counter-satisfiable, or returning an error. Under typical runs, prior to the termination of the program the transitional file is deleted, but it is possible to execute the program in such a way as to leave the file for viewing. This is of course useful for determining why some input fails to execute or returns a questionable result, but it can also be convenient for seeing how these files are structured for execution.

Should a user desire to use the program for their work but find themselves needing axioms which have not been provided, they can easily add TPTP format files to the \texttt{axioms} directory, which contains all the axioms used by the program.

\subsection{Axioms}

Included with the software are three complete axiom sets along with a mostly implemented fourth set. The axiom sets are:
\begin{itemize}
	\item Set theory
	\item Algebra
	\item Absolute geometry
	\item Geometry (Incomplete)
\end{itemize}

\subsubsection{Set Theory}

The ``set theory'' axioms which are contained in the \texttt{axioms} directory are based on the axioms of Zermelo-Fraenkel set theory \cite{setTheory}. This axiomatic system was chosen as it is the basis of the ``set theory'' that would be most familar to a user without knowledge of more advanced mathematical ideas.

Zermelo-Fraenkel (ZF) set theory was described in the early 1900s as a way to axiomatize set theory without falling into traps earlier axiomatizations did, such as Russel's paradox. This paradox asks one to consider the set $S$, which contains all sets that are not members of themselves. Based on the definition of that set, we would expect to see $S$ within $S$, as $S$ did not contain itself. However, now $S$ does contain itself, so we should not allow $S$ in $S$. But once it has been removed, it once again lacks itself so belongs in the set.

To solve this problem, Zermelo described the axiom of Separation, or \textit{Aussonderungsaxiom}: \[\forall A \exists B: \forall x (x \in B \equiv (x \in A \and \phi)) \text{\cite{stanford}.}\]

What this axiom states is that given any set $A$, there exists a set $B$ such that given any set $x$, $x$ will be a member of $B$ if and only if $x$ is in $A$ and a certain formula $\phi$ holds for $x$. An example of a formula would be $x$ is even, meaning our set $B$ contains all even numbers while $A$ could be any set. Because of this, the description of a set would be written \[{x \in z: \phi(x)},\] rather than \[{x:\phi(x)}.\]

While this does not appear to resolve the issue raised by Russel's paradox, it actually invalidates it. Russel's paradox is of the form \[{x: x \notin x},\] which is not valid in our language. We can rephrase this to \[{x \in x: x \notin x},\] which is the most accurate way of stating the paradox in the language of ZF set theory, and reveals very clearly that the paradox makes little sense in this language.

Whle ZF set theory provides an effective understanding of sets, it is also, conveniently, a relatively short axiomatization.

\subsubsection{Algebra}

Algebra is a specific case of what is known as a field, so the axioms used here are the set of field axioms which describe algebraic relations. A field is described by the triple $(F,+,\cdot)$, where $F$ is some set and $(+$, $\cdot)$ are binary operators on $F$. For an operator to be on $F$ means that, when acting on elements of $F$, the only possible result is another element of $F$. Field axioms include associativity of addition and multiplication, existence of identities and inverses, distributive laws, and a distinction between the additive and multiplicative identities. Because algebra is just a type of field, where $+$ is addition and $\cdot$ multiplication, and $F$ the complex numbers, all of these claims apply to it just as they would to algebra. For this reason, one could opt to describe this set of axioms as field axioms rather than algebraic axioms, but due to the use of the \texttt{\$sum()}, \texttt{\$difference()}, and \texttt{\$product()} operators, only algebra is described within this set of axioms.

Through changing these axioms to remove the use of the above functions, a future user could easily adopt this set of axioms for use with any arbitrary field. To completely convert this, the user would also be required to alter a few lines of code from the parser which change any ``sum'' or ``product'' found within the text to ``\$sum'' and ``\$product,'' respectively. Doing so would ensure that no errors occurred as a result of translation from tree to TPTP as a result of this assumption that was made here to enable smooth operation for the user.

\subsubsection{Absolute Geometry}

Absolute geometry refers to non-Euclidean geometry -- Euclid's axioms without the parallel postulate. The parallel postulate states that, should two straight lines be intersected by another, the side on which the sum of the interior angles created by these intersections is less than $180^{\circ}$ will be the side on which the two lines will intersect \cite{absolute}. Geometry in this context refers to the geometry defined by Hilbert's Axioms of Geometry \cite{Hilbert}.

The remaining axioms are rather simple:
\begin{enumerate}
	\item It is possible to form a line segment by connecting two points.
	\item To produce a line, we may extend a line segment.
	\item A circle may be described by its center and its radius.
	\item All right angles are equivalent.
\end{enumerate}

While very simple, absolute geometry stil is relatively powerful in terms of proving ability. The majority of the conjectures which Euclid proves in his work \textit{Elements} may be proven without the use of the parallel postulate. Non-Euclidean geometry, systems of geometry which are formed by rejecting one or more of the claims of Euclid, are researched and many can be proven to be just as consistent as Euclidean geometry. One example of non-Euclidean geometry is elliptic geometry, which is geometry occuring on a sphere, such as the Earth. In this system, given a line $l$ and a point $A$ not on $l$, any line through $A$ will eventually intersect with $l$, i.e., it is not possible to create two parallel lines in this geometry.

\subsubsection{Geometry}

This file was meant to contain a description of Hilbert's axioms in first order form. These axioms are some of the most familar for geometry, being described by Hilbert in 1899 while still being one of the main axiomatizations of Euclidean geometry. Hilbert defines geometry using three basic terms:
\begin{itemize}
	\item point,
	\item line,
	\item plane.
\end{itemize}

\noindent
In addition to these terms, Hilbert defines three basic relations which are possible:
\begin{itemize}
	\item \textit{betweenness}, which describes the relationship between three points,
	\item \textit{containment}, which associates points with lines, points with planes, and lines with planes,
	\item and \textit{congruence}, which describes equivalence between line segments or between angles.
\end{itemize}

By using relations of betweenness and containment, it is possible to describe line segments -- a segement $AB$ is the area of some line $C$ which two points $A$ and $B$ lie on, where any point on $AB$ is between points $A$ and $B$. A triangle is a series of three line segments with shared endpoints, and an angle can be described as the space between two lines before or after an intersection.

Hilbert's axioms consist of a total of twenty claims, so already are clearly more complicated than the axioms of absolute geometry. They discuss angles, parallel lines, continuity, and several other subjects. The file which contains these axioms is not possible to use with the parser, as many of the relations require many variables to be input. At this time, however, the parser assumes that anything input consists of a function and two varables, so an ``incorrect'' number of variables causes the parser to return a faulty conjecture in the best case, and an error in the worst.

\subsubsection{Creating and Using Additional Axioms}

While we do not include instructions on producing new axioms, a user can easily examine the included files to see many of the techniques which are used. The user is thus able to get an idea of how a new set of axioms would be created, both through examination of each claim's format and the file's overall structure.

To use additional axioms, a user need only create the file and put it within the \texttt{axioms} directory. All work of creating a master axiom file is performed by the program. This master file contains all axioms from within the directory, combined into a single file. For this reason, a user may not place their conjecture within their axiom file should they want to use it with the program.

Only TPTP formatted files are appended, so any other files, such as notes or work-in-progress axioms, may exist within the directory as long as they are saved with a different file extension.

The software checks for user determined words within the constructed list of claims to determine which axiom file to use as its base. While initial plans were for all axioms to be formed into a single file at runtime, some errors arose as a result of this. The first issue, which was solved, was a difference in parsing between the statements ``X equals Y'' and ``X is a subset of Y,'' both of which contain their information in a format which the \texttt{NP\_VP()} function, found in section \ref{npvp}, would read. In the subset claim, all pertinent information, $X$, $Y$, and ``subset'' are nouns, while in the other case ``equals'' is a verb. Thus, the information retrieved must be changed, in one case grabbing only nouns while the other retrieves both nouns and a verb.

The second issue was finding that, when running with too long of an axiom file (over 100 lines) all statements would simply be returned as true. This may have been due to some conflict between phrasing of certain axioms, but rather than change what was working, the axioms have instead been kept separate when evaluating conjectures. As end users do not see a difference during use, the change was considered to be a non-issue.
